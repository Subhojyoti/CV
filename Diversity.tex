\documentclass[twoside]{article}
\setlength{\oddsidemargin}{0.25 in}
\setlength{\evensidemargin}{-0.25 in}
\setlength{\topmargin}{-0.6 in}
\setlength{\textwidth}{6.5 in}
\setlength{\textheight}{8.5 in}
\setlength{\headsep}{0.75 in}
\setlength{\parindent}{0 in}
\setlength{\parskip}{0.1 in}

%
% ADD PACKAGES here:
%

\usepackage{amsfonts,graphicx}
%\usepackage{amsmath}
\usepackage{algorithm}
%\usepackage[noend]{algpseudocode}
\usepackage{verbatim}

\usepackage[T1]{fontenc}
\usepackage[utf8]{inputenc}
\usepackage[english]{babel}



\usepackage{macros}


%
% The following commands set up the lecnum (lecture number)
% counter and make various numbering schemes work relative
% to the lecture number.
%
\newcounter{lecnum}
\renewcommand{\thepage}{\arabic{page}}
\renewcommand{\thesection}{\arabic{section}}
\renewcommand{\theequation}{\arabic{equation}}
\renewcommand{\thefigure}{\arabic{figure}}
\renewcommand{\thetable}{\arabic{table}}

%
% The following macro is used to generate the header.
%
\newcommand{\lecture}[4]{
   \pagestyle{myheadings}
   \thispagestyle{plain}
   \newpage
   \setcounter{lecnum}{#1}
   \setcounter{page}{1}
   \noindent
   \begin{center}
   \framebox{
      \vbox{\vspace{2mm}
%    \hbox to 6.28in { {\bf Statement of Purpose
%        \hfill Fall 2018} }
       \vspace{4mm}
       \hbox to 6.28in { {\Large \hfill Diversity Information for Ph.D. application for Fall 2018  \hfill} }
       \vspace{2mm}
       %\begin{center}
       \hbox to 5.28in { {\it \hfill Department: Computing and Mathematical Sciences \hfill Applicant: Subhojyoti Mukherjee} }
       %\end{center}
      \vspace{2mm}}
   }
   \end{center}
   %\markboth{Lecture #1: #2}{Lecture #1: #2}

   
}
%
% Convention for citations is authors' initials followed by the year.
% For example, to cite a paper by Leighton and Maggs you would type
% \cite{LM89}, and to cite a paper by Strassen you would type \cite{S69}.
% (To avoid bibliography problems, for now we redefine the \cite command.)
% Also commands that create a suitable format for the reference list.
\renewcommand{\cite}[1]{[#1]}
\def\beginrefs{\begin{list}%
        {[\arabic{equation}]}{\usecounter{equation}
         \setlength{\leftmargin}{2.0truecm}\setlength{\labelsep}{0.4truecm}%
         \setlength{\labelwidth}{1.6truecm}}}
\def\endrefs{\end{list}}
\def\bibentry#1{\item[\hbox{[#1]}]}

%Use this command for a figure; it puts a figure in wherever you want it.
%usage: \fig{NUMBER}{SPACE-IN-INCHES}{CAPTION}
\newcommand{\fig}[3]{
            \vspace{#2}
            \begin{center}
            Figure #1:~#3
            \end{center}
    }
% Use these for theorems, lemmas, proofs, etc.
%\newtheorem{theorem}{Theorem}
%\newtheorem{lemma}[theorem]{Lemma}
%\newtheorem{proposition}[theorem]{Proposition}
%\newtheorem{claim}[theorem]{Claim}
%\newtheorem{corollary}[theorem]{Corollary}
%\newtheorem{definition}[theorem]{Definition}
%\newenvironment{proof}{{\bf Proof:}}{\hfill\rule{2mm}{2mm}}

% **** IF YOU WANT TO DEFINE ADDITIONAL MACROS FOR YOURSELF, PUT THEM HERE:

%\newcommand\E{\mathbb{E}}


\begin{document}
%FILL IN THE RIGHT INFO.
%\lecture{**LECTURE-NUMBER**}{**DATE**}{**LECTURER**}{**SCRIBE**}
\lecture{1}{A - Title}{Lecturer Name}{scribe-name}

I am a gay Indian student from a middle-income upper caste conservative family in India. It has been particularly difficult for me growing up in my family because of their religious and social belief regarding the subject of sex and sexual orientation and because of this I have passed through periods of mental depression. Moreover, in present-day India, any type of non-penal-vaginal sex is legally punishable by 10 years in prison or a monetary fine or both. There is a huge amount of social stigma, ignorance, apathy, and bigotry present in our Indian society regarding the concept of LGBTQIA+ (Lesbian, Gay, Bisexual, Transgender, Queer, Intersex, Asexual). 
    
    It has been my mission for a long time to create more awareness and sensitivity towards LGBTQIA+ community in the society and especially in my immediate environment. I have been active on this from my bachelor days and in Indian Institute of Technology Madras, my boyfriend and I have started the campus LGBTQIA+ resource/support group called Vannam (meaning color in the Tamil language). We organize meetings between our fellow students from the community and offer support to any traumatized person in the campus who is having difficulty to come to terms with their sexuality apart from organizing occasional event to raise awareness among the student community.  
    
    Stanford will definitely gain from inducting me into its graduate school as not only I will bring more diversity into its fold but I also represent the aspiration of an oppressed sexual minority who has to struggle against social stigma and oppression to reach the point where I currently belong.

%I am a gay Indian student from a middle-income upper caste conservative family in India. It has been particularly difficult for me growing up in my family because of their religious and social belief regarding the subject of sexual orientation and because of this I have passed through periods of mental depression. Also, in India, any type of non-penal-vaginal sex is legally punishable by 10 years in prison or a monetary fine or both. In IIT Madras, my boyfriend and I have started the campus LGBTQIA support group called Vannam (meaning color in Tamil) to provide support to LGBTQIA students. I represent the aspiration of an oppressed sexual minority and will increase the diversity of Stanford.

\end{document}



